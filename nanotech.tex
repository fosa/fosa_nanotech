% coding:utf-8

%----------------------------------------
%FOSAPHY, a LaTeX-Code for a summary of basic physics
%Copyright (C) 2013, Mario Felder, Michael Fallegger

%This program is free software; you can redistribute it and/or
%modify it under the terms of the GNU General Public License
%as published by the Free Software Foundation; either version 2
%of the License, or (at your option) any later version.

%This program is distributed in the hope that it will be useful,
%but WITHOUT ANY WARRANTY; without even the implied warranty of
%MERCHANTABILITY or FITNESS FOR A PARTICULAR PURPOSE.  See the
%GNU General Public License for more details.
%----------------------------------------

\chapter{Nanotechnologie}

%\section{Konstanten}
%\[
%\boxed{\begin{aligned}	
%		&\text{Avogadrozahl} \\
%		&N_{A} = 6.00221 \cdot 10^{23}\text{ Teilchen}\\
%		\\
%		&\text{Elektron Masse}\\
%		&m_e = 9.1 \cdot 10^{-31}\text{ kg}\\
%		\\
%		&\text{Proton Masse}\\
%		&m_p = 1.7 \cdot 10^{-27}\text{ kg}\\
%		\\
%		&\text{Elementarladung}\\
%		&e = 1.6 \cdot 10^{-19}\text{ C}\\
%		\\
%		&\text{Elektronenvolt}\\
%		&ev = 1.6 \cdot 10^{-19}\text{ J}\\
%		\\
%		&\text{Planck Konstante}\\
%		&h = 6.63 \cdot 10^{-34}\text{ Js}\\
%		&$\hbar=1.06 \cdot 10^{-34}$\text{ Js}
%	\end{aligned}}	\]
%

\section{Kräftevergleich}
Es wirkt die Schwerkraft nach unten un die elektrostatische Kraft nach oben.\\
\[
	F_{el}=\frac{1}{4\pi\epsilon_0} \cdot \frac{q_1 \cdot q_2}{r^2}
\]
\begin{footnotesize}
	$\epsilon_0 = 8.85\cdot 10^{-12}$
\end{footnotesize}
\\
\\
Für Teilchen unterhalb einer kritischen Grösse (Radius $r_c$) sind Gewichts- und Auftriebskräfte nicht mehr relevant.
\[
	r_c=\left( \frac{3 k_b\cdot T}{4\pi\cdot\Delta \rho \cdot a}\right) ^{1/4}
\]
\\
\begin{footnotesize}
	$T:$ Temperatur in Kelvin\\
	$a:$ Beschleunigung der ext. Kraft
\end{footnotesize}
\\
\\
Grafik von Mariooooooooo
\\
\section{Atome und Nanostrukturen}
Anzahl Atome $N$ in einem reinen Material mit Volumen V:
\[
	N=\frac{\rho\cdot N_A}{m_{mol}} \cdot V
\]
\\
\\
Gemittelter Atomabstand a:
\[
	a \approx \left( \frac{m_{mol}}{\rho\cdot N_A}\right) ^{1/3}
\]
\\
\\
Anzahl Oberflächenatome N in einem reinen Material mit Fläche $A$:
\[
	N_s=\left( \frac{\rho\cdot N_A}{m_{mol}}\right) ^{2/3}\cdot A
\]
\\
\begin{footnotesize}
	$N_A=6.02\cdot 10^{23} mol$
\end{footnotesize}
\\
\\
\section{Schmelztemperatur}
Liegt der Durchmesser von Nanoteilchen unterhalb von 50nm, wird die Schmelztemperatur reduziert. Dieser Effekt ist auf die grosse Oberfläche im Verhältnis zum Volumen zurückzuführen.\\
\\
Änderung der Schmelztemperatur:
\[
	\Delta T_m=\frac{4\cdot \gamma_{sl}}{d \cdot \rho \cdot L_f} \cdot Tb
\]
\\
\begin{footnotesize}
	$T_b$: Schmelztemperatur (b=bulk)\\
	$d$: Durchmesser\\
	$\rho$: Dichte\\
	$\gamma_{sl}$: solid-liquid interface energy\\
	$L_f$: Latente Schmelzwärme\\
\end{footnotesize}