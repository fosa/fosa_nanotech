% coding:utf-8

%----------------------------------------
%FOSAPHY, a LaTeX-Code for a summary of nanotechnology
%Copyright (C) 2014, Mario Felder, Michael Fallegger

%This program is free software; you can redistribute it and/or
%modify it under the terms of the GNU General Public License
%as published by the Free Software Foundation; either version 2
%of the License, or (at your option) any later version.

%This program is distributed in the hope that it will be useful,
%but WITHOUT ANY WARRANTY; without even the implied warranty of
%MERCHANTABILITY or FITNESS FOR A PARTICULAR PURPOSE.  See the
%GNU General Public License for more details.
%----------------------------------------

\chapter{Quantenphysik}

\section{Wellenlänge eines Teilchens}
\[\boxed{
	\lambda = \frac{h}{p} = \frac{h}{m \cdot v}
}\]
\\
\begin{footnotesize}
	$h = 6.63 \cdot 10^{-34}Js$: Plancksches Wirkungsquantum 
\end{footnotesize}
\\

\section{Licht als Teilchen}
Photoenergie:
\[\boxed{
	E_{ph} = h \cdot f = \hbar \cdot \omega = \frac{h \cdot c}{\lambda}
}\]
\\
Photompuls:
\[\boxed{
	p_{ph} = \hbar \cdot k = \frac{h \cdot f}{c} = \frac{h}{\lambda}
}\]

\section{Unschärferelation}
\[\boxed{
	\Delta x \cdot \Delta p_x \ge \hbar
}\]
\\
Unschärfe für Zeit und Energie:
\[\boxed{
	\Delta E \cdot \Delta t \ge h
}\]
\\
Unschärfe für Zeit und Frequenz:
\[\boxed{
	\Delta f \cdot \Delta t \ge 1 \qquad \Leftrightarrow \qquad \frac{c}{\lambda^2} \Delta \lambda \cdot \Delta t \ge 1
}\]
\\

\section{Wellenfunktion}
\[\boxed{
	\rho(x,t) = \left| \psi(x,t) \right|^2 
}\]
\\
Schrödingergleichung:
\[\boxed{
	\im h \pdifrac{}{t} \psi(x,t) = H \psi(x,t)
}\]
\[
	H = \underbrace{- \frac{h^2}{2m} \cdot \pdifrac{^2}{x^2}}_{\text{kinetisch}} + \underbrace{V(x,t)}_{\text{potentiell}}
\]
\\
mit:
\[
	\psi(x,t) = \e^{-\im \frac{E \cdot t}{h}} \cdot \varphi(x)
\]
folgt:
\[
	E \cdot \varphi(x) = - \frac{h^2}{2m} \cdot \pdifrac{^2 \varphi(x)}{x^2} + V(x) \cdot \varphi(x)
\]
\\
Wahrscheinlichkeit dass ein Teilchen an einer bestimmten Position $x$ ist:
\[\boxed{
	\rho(x) = \left| \psi(x,t) \right|^2 = \left| \varphi(x) \right|^2
}\]
\[
	\int \rho(x) \di x = 1
\]
\\
\begin{footnotesize}
	$\rho(x) \left[ \frac{1}{m} \right]$\\
	$\varphi(x) \left[ \frac{1}{m^2} \right]$\\
	$\psi(x) \left[ \frac{1}{m^2} \right]$
\end{footnotesize}
\\

\section{Elektron in einem Potentialtopf}
innerhalb $0 < x < L$:
\[\boxed{
	E \cdot \varphi(x) = - \frac{h^2}{2m} \cdot \pdifrac{^2 \varphi(x)}{x^2}
}\]
\\
Lösungen der DGL:
\[\boxed{
	\varphi_n(x) = \sqrt{\frac{2}{L}} \sin(k_n \cdot x)
}\]
\[\boxed{
	k_n = n \cdot \frac{\pi}{L}
}\]
\begin{footnotesize}
	$n = 1,2,\dots$: Quantisierungszahl
\end{footnotesize}
\[\boxed{
	E_n = \frac{h^2 k_n^2}{2m} \propto \frac{n^2}{L^2}
}\]

\subsection{Endlicher Potentialtopf}
Eindringtiefe:
\[\boxed{
	d_n = \frac{h}{\sqrt{2m(U_0 - E_n)}}
}\]

\subsection{Tunnelbarriere}
Transmissionswahrscheinlichkeit:
\[\boxed{
	T \approx \e^{-\frac{2L}{d}}
}\]
\[\boxed{
	d = \frac{h}{\sqrt{2m(U_0-E)}}
}\]